\documentclass[11pt, a4paper]{article}
\usepackage{amsmath, amsfonts, dsfont, booktabs, graphicx, natbib, a4wide, times, microtype, hyperref}
\newcommand{\E}{\ensuremath{{\mathbb E}}} % expected value
\def\func#1{\mathop{\rm #1}}
\begin{document}
\title{Solution to Exercise 3}
\author{Simon A.\ Broda}
\date{}
\maketitle
\begin{enumerate}
\item
\begin{enumerate}
\item We clearly see that unless $|\phi_1|$ approaches 1, the process is stationary; the time series plot looks mean-reverting, and the sample autocorrelations decay exponentially as they should. We also see that $\bar y$ is close to $\E[Y_t]=\alpha/(1-\phi_1)$, and that $s^2_y$ is close to $\func{var}[Y_t]=\sigma^2/(1-\phi_1^2)$.
\item If $\phi_1=1$, we have a random walk, and $\alpha$ becomes the drift: $\E[Y_t]=Y_0+\alpha\cdot t$.
\item See Jupyter notebook.
\end{enumerate}
\item See Jupyter notebook.

\item
\begin{enumerate}
\item By repeatedly plugging in,
\begin{eqnarray*}
Y_t&=&\alpha+Y_{t-1}+U_t\\
   &=&\alpha+(\alpha +Y_{t-2}+U_{t-1})+U_t\\
  &\vdots&\\
&=&Y_0+\alpha\cdot t+\sum_{s=1}^t U_s,
\end{eqnarray*}
so that
\[
\E[Y_t]=Y_0+ \alpha\cdot t,
\]
because white noise has expectation zero. The derivation of the variance is the same as for the case without drift from last week and thus omitted here.
\item The previous question shows that the random walk with drift is not stationary, because its mean and variance change over time. For it to be I(1), its first difference $\Delta Y_t$ should be stationary.
We immediately se that $\Delta Y_t=Y_t-Y_{t-1}=(\alpha+Y_{t-1}+U_t)-Y_{t-1}=\alpha + U_t$. This is just white noise plus a constant, which is stationary.
\item
Since $\{U_t\}$ is white noise, $U_t$  is uncorrelated with $Y_{t-1}$, so
\begin{align*}
\mathrm{var}(Y_t) &= \func{var}\left(\alpha+\phi_1Y_{t-1}+U_{t}\right)\\
&= \phi_1^2 \mathrm{var}(Y_{t-1}) + \mathrm{var}(U_t) + 2
\phi_1 \mathrm{cov}(Y_{t-1},U_{t})
&= \phi_1^2 \mathrm{var}(Y_{t}) +
\sigma^2,
\end{align*}
where the final equality holds because $Y_t$ is stationary, which implies that $\func{var}(Y_t)=\func{var}(Y_{t-1})$. Thus, if and only if $|\phi_1 |<1$,
\begin{equation*}
\mathrm{var}(Y_t) = \frac{\sigma^2}{1-\phi_1^2}.
\end{equation*}
Note that $\mathrm{var}(Y_t) > \mathrm{var}(Y_{t-1})$ if $|\phi_1 |\ge 1$, i.e., the variance grows without bounds in that case.
\item \textbf{Optional}: For the MA(1) process
\begin{equation*}
Y_t=\alpha + U_t + \theta_1 U_{t-1},
\end{equation*}
we have that
\begin{align*}
\E[Y_t]&=\E[\alpha + U_t + \theta_1 U_{t-1}]\\
&=\alpha + \E[U_t] + \theta_1 \E[U_{t-1}]\\
&=\alpha.
\end{align*}
For the variance,
\begin{align*}
\gamma_0&=\func{var}(Y_t)=\func{var}(\alpha + U_t + \theta_1 U_{t-1})\\
&=\func{var}(U_t + \theta_1 U_{t-1})\\
&=\func{var}(U_t) + \theta_1^2\func{var}( U_{t-1})+2\theta_1\func{cov}(U_t, U_{t-1})\\
&=\sigma^2+\theta_1^2\sigma^2+0\\
&=\sigma^2(1+\theta_1^2).
\end{align*}
For the first autocovariance,
\begin{align*}
\gamma_1&=\func{cov}(Y_t, Y_{t-1})\\
&=\func{cov}(\alpha + U_t + \theta_1 U_{t-1}, \alpha + U_{t-1} + \theta_1 U_{t-2})\tag{$\dagger$}\label{ma}\\
&=\func{cov}(\theta_1 U_{t-1},U_{t-1})\\
\intertext{because white noise is uncorrelated. Hence}
\gamma_1&=\theta_1\func{cov}( U_{t-1},U_{t-1})\\
&=\theta_1\func{var}(U_{t-1})\\
&=\theta_1\sigma^2.
\end{align*}
Higher order autocorrelations will be zero, because there will no common $U_t$ terms in \eqref{ma}.
Plugging these into the definition of the ACF, we have
\[
\tau_1=\frac{\gamma_1}{\gamma_0}=\frac{\theta_1\sigma^2}{\sigma^2(1+\theta_1^2)}=\frac{\theta_1}{1+\theta_1^2}.
\]


\item \textbf{Optional}:
The ACF is obtained by repeatedly
substituting $Y_{t-i} = \phi_1 Y_{t-i-1} + \alpha + U_{t-i}$:
\begin{eqnarray}
Y_t &=&\phi_1 Y_{t-1}+ \alpha + U_t  \notag \\
&=&\phi_1 ^2Y_{t-2} + \phi_1 (\alpha + U_{t-1}) + \alpha + U_t  \notag \\
&=&\phi_1 ^3 Y_{t-3}+ \phi_1 ^2 (\alpha + U_{t-2} )+ \phi_1 (\alpha + U_{t-1})+
\alpha + U_t  \notag \\
&\vdots &  \notag \\
&=&\phi_1 ^k Y_{t-k}+\sum_{i=0}^{k-1}\phi_1 ^i \alpha + \sum_{i=0}^{k-1}\phi_1
^i U_{t-i}.  \label{subst}
\end{eqnarray}

Therefore,
\begin{align*}
\gamma_k &= \mathrm{cov}(Y_t,Y_{t-k}) = \phi_1 ^k \mathrm{cov}%
(Y_{t-k},Y_{t-k}) + \sum_{i=0}^{k-1}\phi_1 ^i \mathrm{cov}(U_{t-i},Y_{t-k})\\
&= \phi_1 ^k \mathrm{var}(Y_{t-k}),
\end{align*}
so that
\begin{equation*}
\tau_k = \frac{\gamma_k}{\gamma_0} = \phi_1 ^k.
\end{equation*}

\end{enumerate}
\end{enumerate}
\end{document} 