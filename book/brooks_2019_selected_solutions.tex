
\documentclass[11pt,a4paper]{article}
%%%%%%%%%%%%%%%%%%%%%%%%%%%%%%%%%%%%%%%%%%%%%%%%%%%%%%%%%%%%%%%%%%%%%%%%%%%%%%%%%%%%%%%%%%%%%%%%%%%%%%%%%%%%%%%%%%%%%%%%%%%%%%%%%%%%%%%%%%%%%%%%%%%%%%%%%%%%%%%%%%%%%%%%%%%%%%%%%%%%%%%%%%%%%%%%%%%%%%%%%%%%%%%%%%%%%%%%%%%%%%%%%%%%%%%%%%%%%%%%%%%%%%%%%%%%
\usepackage{a4wide}
\usepackage{times}
\usepackage{amsmath, amsfonts}
%\usepackage{amsmath}

%\setcounter{MaxMatrixCols}{10}
%TCIDATA{OutputFilter=LATEX.DLL}
%TCIDATA{Version=5.00.0.2552}
%TCIDATA{<META NAME="SaveForMode" CONTENT="1">}
%TCIDATA{Created=Thursday, December 16, 2004 13:22:33}
%TCIDATA{LastRevised=Thursday, December 16, 2004 13:49:59}
%TCIDATA{<META NAME="GraphicsSave" CONTENT="32">}
%TCIDATA{<META NAME="DocumentShell" CONTENT="Articles\myart">}
%TCIDATA{CSTFile=myart.cst}



\begin{document}


\section*{Solutions to selected problems from Brooks (2019)}
\subsection*{Chapter 6 (p.\ 381--386): Autocorrelation, stationarity, ARMA models.}


\begin{itemize}
\item[6.3]
\begin{enumerate}
\item[(a)] The first two models are examples of AR(1) models, with AR parameter $\phi_1$ equal to $1$ (random walk) and $0.5$ (stationary AR(1) process), respectively. The third model is an MA(1).
\item[(b)] For $|\phi_1| < 1$ the acf of the AR(1) model decays geometrically
(it is equal to $\tau_k = \phi_1^k$).  In the case $\phi_1=1$ the theoretical
acf does not even exist (in that case the time series process is a random walk,
which is nonstationary). However, if you would nevertheless estimate the
acf, you would find that the estimator of $\tau_k$ decays only very slowly
with $k$, and seemingly linearly. For an MA(1), the ACF drops to zero after the first lag.
\item[(c)] The random walk is most reasonable as a model for stock prices or stock indices. For $\phi<1$ price changes would be predictable. However, traders
exploiting this would drive the price up to the point where the price change would be unpredictable again. Hence in practice only very little (or no)
predictability should be expected.
\end{enumerate}
\item[6.6] We end up with the following values for
each criterion and for each model order, with an asterisk denoting the smallest
value of the information criterion in each case. SBIC stands for Schwarz Bayesian Criterion, also known as SIC or BIC.
\begin{center}
\begin{tabular}{ccclll}
ARMA order $(p,q)$ & $k=p+q-1$ & $\log \hat \sigma^2$ & AIC & SBIC \\
 $(0,0)$ & $1$ & $0.932$ & $0.942$ & $0.944$ \\
 $(1,0)$ & $2$ & $0.864$ & $0.884$ & $0.887$ \\
 $(0,1)$ & $2$ & $0.902$ & $0.922$ & $0.925$ \\
 $(1,1)$ & $3$ & $0.836$ & $0.866$ & $0.870$ \\
 $(2,1)$ & $4$ & $0.801$ & $0.841$ & $0.847$ \\
 $(1,2)$ & $4$ & $0.821$ & $0.861$ & $0.867$ \\
 $(2,2)$ & $5$ & $0.789$ & $0.839$ & $0.846$ \\
 $(3,2)$ & $6$ & $0.773$ & $0.833^*$ & $0.842^*$ \\
 $(2,3)$ & $6$ & $0.782$ & $0.842$ & $0.851$ \\
 $(3,3)$ & $7$ & $0.764$ & $0.834$ & $0.844$ \\
\end{tabular}
\end{center}
The result is pretty clear: both SBIC and AIC say that the appropriate model is an ARMA$(3,2)$.

In general, the AIC and SBIC need not agree on the optimal model. Depending
on the aim of the modeller, either the AIC or SBIC can be used. For constructing a model for prediction (where the aim is a small forecast error) the AIC is
more suitable. The SBIC should be used when you want to obtain the
true ARMA($p,q$) orders of the process that generated the data. The assumption
that the data are from a finite order ARMA($p,q$) model is a rather strong one
to start with, so in most applications recovering the true model order is
not the primary goal of the modelling exercise.
\item[6.7] We could perform the Ljung-Box test on the residuals of the
estimated models to see if there was any linear dependence left unaccounted for
by our postulated models.

Another test of the models’ adequacy that we could
use is to leave out some of the observations at the identification and
estimation stage, and attempt to construct out of sample forecasts for these.
For example, if we have $2000$ observations, we may use only $1800$ of them to
identify and estimate the models, and leave the remaining $200$ for
construction of forecasts. We would then prefer the model that gave the most
accurate forecasts.
\item[6.8]  This is not true in general. Yes, we do want to form a model which
``fit'' the data as well as possible. But in most financial series, there is a
substantial amount of ``noise''. This can be interpreted as a number of random
events that are unlikely to be repeated in any forecastable way. We want to fit
a model to the data which will be able to ``generalise''. In other words, we
want a model which fits to features of the data which will be replicated in
future; we do not want to fit to sample-specific noise.

This is why we need
the concept of ``parsimony'' --- fitting the smallest possible model to the data.
Otherwise we may get a great fit to the data in sample, but any use of the
model for forecasts could yield terrible results.

Another important point is
that the larger the number of estimated parameters (i.e. the more variables we
have), then the smaller will be the number of degrees of freedom, and this will
imply that coefficient standard errors will be larger than they would otherwise
 have been. This could lead to a loss of power in hypothesis tests, and
variables that would otherwise have been significant are now insignificant.
\item[6.9]
\begin{itemize}
\item[(a)] We class an autocorrelation coefficient or partial autocorrelation
coefficient as  significant if it exceeds $\pm 1.96 \frac{1}{\sqrt{T}}=0.196$ (Bartlett bands). Under this rule, the sample autocorrelation functions (sacfs) at lag
 $1$ and $4$ are
significant, and the spacfs at lag $1$, $2$, $3$, $4$ and $5$ are all significant. This
clearly looks like the data are consistent with a first order moving average
process since all but the first acfs are not significant (the significant lag
$4$ acf is a typical wrinkle that one might expect with real data and should
probably be ignored), and the pacf has a slowly declining structure.
\item[(b)]
$$
 Q^* = T(T+2) \sum_{k=1}^3 \frac{\hat \tau^2_k}{T-k} = 100*102*\left(
\frac{(0.42)^2}{99}+\frac{(0.104)^2}{98}+\frac{(0.032)^2}{97}
\right) = 19.408
$$
Under $H_0$ (no autocorrelation) $Q^* \sim \chi^2_3$. the $5\%$ critical
value is (Table 3 in reader) $7.815$. The observed value exceeds the critical
value, so we reject the null hypothesis of no serial correlation.
\end{itemize}
\item[6.10]
The forecasts are the conditional expectations given the information
available at time $t-1$, $\Omega_{t-1} = (y_{t-1}, y_{t-2}, \ldots ; u_{t-1}, u_{t-2}, \ldots)$.
\begin{itemize}
\item[(a)]
$$
\begin{array}{rcl}
 f_{t-1,1} = E(y_t|\Omega_{t-1}) & = & 0.036 +0.69 E(y_{t-1}|\Omega_{t-1}) + 0.42\times E(u_{t-1}| \Omega_{t-1}) + E(u_t| \Omega_{t-1}) \\
 & = & 0.036 +0.69 \times y_{t-1} + 0.42\times u_{t-1} + 0 \\
 & = & 0.036 +0.69 \times 3.4 + 0.42\times -1.3 \\
& = & 1.836
\end{array}
$$
$$
\begin{array}{rcl}
 f_{t-1,2} = E(y_{t+1}|\Omega_{t-1}) & = & 0.036 +0.69 E(y_{t}|\Omega_{t-1}) + 0.42\times E(u_{t}| \Omega_{t-1}) + E(u_{t+1}| \Omega_{t-1}) \\
 & = & 0.036 +0.69 \times E(y_{t}|\Omega_{t-1}) + 0 \\
 & = & 0.036 +0.69 \times 1.836  \\
&\approx & 1.3028
\end{array}
$$
$$
\begin{array}{rcl}
 f_{t-1,3} = E(y_{t+2}|\Omega_{t-1}) & = & 0.036 +0.69 E(y_{t+1}|\Omega_{t-1})\\ && +\; 0.42\times E(u_{t+1}| \Omega_{t-1}) + E(u_{t+2}| \Omega_{t-1}) \\
 & = & 0.036 +0.69 \times E(y_{t+1}|\Omega_{t-1}) + 0 \\
 & \approx & 0.036 +0.69 \times 1.3028  \\
& \approx & 0.935.
\end{array}
$$
\item[(b)] Given the forecasts and the actual value, it is very easy to
calculate the MSE by plugging the numbers in to the relevant formula, which in
this case is
$$
MSE =
\frac{1}{N} \sum_{n=1}^N \left( x_{t -1+ n} - f_{t -1, n} \right)^2
$$
if we are making $N$ forecasts. Then the MSE is given by
$$
\begin{array}{rcl}
MSE &  =  & (1.836 + 0.032)^2 + (1.302 - 0.961)^2 + (0.935 - 0.203)^2  \\
& = & 3.489 + 0.116 + 0.536 =  4.141.
\end{array}
$$
Notice also that $84\%$ of the total sum of squared errors is coming from the error
in the first forecast. Thus error measures can be driven by one or two times
when the model fits very badly. For example, if the forecast period includes
a stock market crash, this can lead the mean squared error to be much
bigger than it would have been if the crash observations were not included.
This point needs to be considered whenever forecasting models are evaluated.
An idea of whether this is a problem in a given situation can be gained by
plotting the forecast errors over time.
\item[(c)] This question is much simpler to answer than it looks! In fact, the
 inclusion of the smoothing coefficient is a ``red herring'' - i.e. a piece of
misleading and useless information. The correct approach is to say that if we
believe that the exponential smoothing model is appropriate, then all useful
information will have already been used in the calculation of the current
smoothed value (which will of course have used the smoothing coefficient in
its calculation). Thus the three forecasts are all $0.0305$.

More formally, note that the exponential smoothing forecast applies in the case of
an ARIMA($0$,$1$,$1$) model. Thus
$$
 f_{t-1,1} = E(y_t| \Omega_{t-1}) = S_{t-1} = 0.0305.
$$
The two-step ahead forecast is
$$
 f_{t-1,2} = E(y_{t+1}|\Omega_{t-1}) = E(S_t|\Omega_{t-1}) = \alpha E(y_t|\Omega_{t-1}) + (1-\alpha) S_{t-1} = S_{t-1}.
$$
The 2-step-ahead forecast is the same as the one-step ahead forecast.
In a similar way it can be  shown that $f_{t-1,k} = f_{t-1,k-1}$
for $k=3,4,\ldots$.
\item[(d)] The solution is to work out the mean squared error for the exponential smoothing model. The calculation is
$$
\begin{array}{rcl}
MSE & = &
\frac{1}{3} \left((0.0305 - -0.032)^2 + (0.0305 - 0.961)^2 + (0.0305 - 0.203)^2 \right) \\
& = & (0.0039 + 0.8658 + 0.0298) = 0.2998.
\end{array}
$$
Therefore, we conclude that since the mean squared error is smaller for the
exponential smoothing model than the Box Jenkins model, the former produces
the more accurate forecasts. We should, however, bear in mind that the
question of accuracy was determined using only 3 forecasts, which would be
insufficient in a real application.
\end{itemize}
\item[6.11]
\begin{itemize}
\item[(a)]
The shapes of the acf and pacf are perhaps best summarised in a table: \\
\begin{tabular}{c|cc}
\hline
Process & acf & pacf\\
\hline
White noise & No significant coefficients & No significant
coefficients \\
AR($2$) & Geometrically declining or damped & First 2 pacf coefficients
significant, \\
 & sinusoid & others insignificant \\
MA($1$) & First acf coefficient
significant, all & Geometrically declining or\\
 & others insignificant & damped sinusoid \\
ARMA($2$,$1$) & Geometrically declining or damped & Geometrically
declining or damped\\
 & sinusoid & sinusoid\\
\hline
\end{tabular}
A couple of further points are
worth noting. First, it is not possible to tell what the signs of the
coefficients for the acf or pacf would be for the last three processes, since
that would depend on the signs of the coefficients of the processes. Second,
for mixed processes, the AR part dominates from the point of view of acf
calculation, while the MA part dominates for pacf calculation.
\item[(b)]
The important point here is to focus on the MA part of the model and to
ignore the AR dynamics. The characteristic equation would be ($1+0.42z = 0$) The
root of this equation is $-1/0.42 = -2.38$, which lies outside the unit circle,
and therefore the MA part of the model is invertible.
\item[(c)]
Since no values for
 the series $y$ or the lagged residuals are given, the answers should be stated
in terms of $y$ and of $u$. Assuming that information is available up to and
including time $t$, the $1$-step ahead forecast would be for time
$t+1$, the $2$-step
ahead for time $t+2$ and so on. A useful first step would be to write the model
out for $y$ at times $t+1$, $t+2$, $t+3$, $t+4$:
$$
y_{t +1} = 0.21 + 1.32 y_t + 0.58 u_t + u_{t +1}
$$
$$
y_{t + 2} = 0.21 + 1.32 y_{t +1} +  0.58 u_{t +1} + u_{t + 2}
$$
$$
 y_{t +3} = 0.21 + 1.32 y_{t + 2} + 0.58 u_{t + 2} + u_{t +3}
$$
$$
y_{t + 4} = 0.21 + 1.32 y_{t +3} + 0.58 u_{t +3} + u_{t + 4}
$$
The $1$-step ahead forecast
would simply be the conditional expectation of y for time $t+1$ made at time
$t$.
Denoting the $1$-step ahead forecast made at time $t$ as $f_{t,1}$, the $2$-step ahead
forecast made at time $t$ as $f_{t,2}$ and so on. Denoting
conditional expectation based on the information, $\Omega_t$, available
at time $t$ by  $E_{t}$, we have
$$
\begin{array}{rcl}
E( y_{t +1}| y_{t}, y_{t -1} , \ldots) = f_{t,1} = E_t [ y_{t +1} ] & = & E_t [0.021 + 1.32 y_t + 0.58u_t + u_{t +1} ] \\
 & = & = 0.21 + 1.32 y_t + 0.58u_t
\end{array}
$$
since $E_t[u_{t+1}]=0$.
The 2-step ahead forecast would be given by
$$
\begin{array}{rcl}
E ( y_{t + 2}| y_t , y_{t -1}, \ldots) = f_{t,2} & = & E_t [ y_{t + 2} ]
= E_t [0.21 + 1.32 y_{t +1} + 0.58u_{t +1} + u_{t + 2} ] \\
& = & 0.21 + 1.32 f_{t,1}
\end{array}
$$
since $E_t[u_{t+1}]=0$ and $E_t[u_{t+2}]=0$. Thus, beyond $1$-step ahead, the
 MA($1$) part of the model disappears from the forecast and only the
autoregressive part remains. Although we do not know $y_{t+1}$, its expected
value is the $1$-step ahead forecast that was made at the first stage, $f_{t,1}$.
The 3-step ahead forecast would be given by
$$
\begin{array}{rcl}
E ( y_{t + 3}| y_t, y_{t -1}, \ldots) = f_{t,3} = E_t [ y_{t + 3} ] & = & E_t [0.21 + 1.32 y_{t + 2} + 0.42 u_{t + 2} + u_{t + 3} ] \\
&= & 0.21 + 1.32 f_{t,2}
\end{array}
$$
and the 4-step ahead by
$$
\begin{array}{rcl}
E ( y_{t + 4}| y_t , y_{t -1}, \ldots) = f_{t,4} = E_t [ y_{t + 4} ] & = & E_t [0.21 + 1.32 y_{t + 3} + 0.42 u_{t + 3} + u_{t + 4} ] \\
& = & 0.21 + 1.32 f_{t ,3}.
\end{array}
$$

\item[(e)]
Moving average and ARMA models cannot be estimated using OLS – they are
usually estimated by maximum likelihood. Autoregressive models can be
estimated using OLS or maximum likelihood. Pure autoregressive models contain
only lagged values of observed quantities on the RHS, and therefore, the lags
of the dependent variable can be used just like any other regressors. However,
in the context of MA and mixed models, the lagged values of the error term
that occur on the RHS are not known a priori. Hence, these quantities are
replaced by the residuals, which are not available until after the model has
been estimated. But equally, these residuals are required in order to be able
to estimate the model parameters. Maximum likelihood essentially works around
this by calculating the values of the coefficients and the residuals at the
same time. Maximum likelihood involves selecting the most likely values of the
 parameters given the actual data sample, and given an assumed statistical
distribution for the errors.
\end{itemize}
\item[6.12]
\begin{itemize}
\item[(a)]
 Some of the stylised differences between the typical characteristics of
macroeconomic and financial data are as follows.
One important difference is the frequency with which financial
asset return time series and other quantities in finance can be recorded. This
is of particular relevance for regressions with time series, since it is
usually a requirement that all of the time-series data series used in
estimating a given model must be of the same frequency. Thus, if, for example,
we wanted to build a model for forecasting hourly changes in exchange rates,
it would be difficult to set up a structural model containing macroeconomic
explanatory variables since the macroeconomic variables are likely to be
measured on a quarterly or at best monthly basis. This gives a motivation for
using pure time-series approaches (e.g. ARMA models), rather than structural
formulations with separate explanatory variables.
It is also often of particular interest to produce forecasts of financial
variables in real time. Producing forecasts from pure time-series models is
usually simply an exercise in iterating with conditional expectations. But
producing forecasts from structural models is considerably more difficult, and
would usually require the production of forecasts for the structural variables
as well.
\item[(b)]  A simple``rule of thumb'' for determining whether autocorrelation
coefficients and partial autocorrelation coefficients are statistically
significant at the the 5\% level is to compare their absolute value to $1.96/\sqrt{T}$
In this case, T = 500,
so a particular coefficient would be deemed significant if it is larger than
$0.088$ or smaller than $–0.088$. On this basis, the autocorrelation
coefficients at lags 1 and 5 and the partial autocorrelation coefficients at
lags $1$, $2$, and $3$ would be classed as significant. The formulae for the
Box-Pierce and the Ljung-Box test statistics are respectively
$$
Q = T \sum_{k=1}^m \hat \tau_k^2
$$
and
$$
Q^* = T(T+2) \sum_{k=1}^m \frac{\hat \tau_k^2}{T-k}.
$$
In this instance, the statistics would be calculated respectively as
$$
Q = 500 × \left[0.307^2 + 0.013^2 + 0.086^2 + 0.0312^2 + 0.197^2\right] = 70.79
$$
and
$$
Q^* = 500\times 502 \times \left[\frac{0.307^2}{500-1} + \frac{0.013^2}{500-2} + \frac{0.086^2}{500-3} + \frac{0.0312^2}{500-4} + \frac{0.197^2}{500-5}\right] = 71.39
$$
The test statistics will both follow a $\chi^2$ distribution with $5$ degrees
of freedom (the number of autocorrelation coefficients being used in the test).
 The critical values are $11.07$ and $15.09$ at $5\%$ and $1\%$ respectively.
Clearly,
the null hypothesis that the first 5 autocorrelation coefficients are jointly
zero is resoundingly rejected
\item[(c)] Bartlett significance bands: $\pm 1.96/\sqrt{500} = \pm 0.088$. Setting aside the lag 5 autocorrelation coefficient, the pattern in
the table is for the autocorrelation coefficient to only be significant at lag
1 and then to fall rapidly to values close to zero, while the partial
autocorrelation coefficients appear to fall much more slowly as the lag length
increases. These characteristics would lead us to think that an appropriate
model for this series is an MA(1). Of course, the autocorrelation coefficient
at lag 5 is an anomaly that does not fit in with the pattern of the rest of
the coefficients. But such a result would be typical of a real data series (as
opposed to a simulated data series that would have a much cleaner structure).
This serves to illustrate that when econometrics is used for the analysis of
real data, the data generating process was almost certainly not any of the
models in the ARMA family. So all we are trying to do is to find a model that
best describes the features of the data to hand. As one econometrician put it,
all models are wrong, but some are useful!
\item[(e)] %The methods are overfitting and residual diagnostics. Overfitting
%involves selecting a deliberately larger model than the proposed one, and
%examining the statistical significances of the additional parameters. If the
%additional parameters are statistically insignificant, then the originally
%postulated model is deemed acceptable. The larger model would usually involve
%the addition of one extra MA term and one extra AR term. Thus it would be
%sensible to try an ARMA(1,2) in the context of Model A, and an ARMA(3,1) in
%the context of Model B.
We have only discussed one method: residual diagnostics. This would involve examining the acf
and pacf of the residuals from the estimated model. If the residuals showed
any ``action'', that is, if any of the acf or pacf coefficients showed
statistical significance, this would suggest that the original model was
inadequate. It is worth noting that these two model evaluation
procedures would only indicate a model that was too small. If the model were
too large, i.e. it had superfluous terms, these procedures would deem the
model adequate.
\end{itemize}
\end{itemize}
\newpage
\subsection*{Chapter 8 (p.\ 492--496): (Co-)Integration}
\begin{itemize}
\item[8.2]
\begin{enumerate}
\item[(a)] The null hypothesis is $H_0$: $\Psi = 0$ while the alternative
hypothesis is $H_a$: $\Psi < 0$.
\item[(b)] The null hypothesis should be rejected if the ratio
$
\frac{\widehat \Psi}{SE(\widehat \Psi)}
 \leq \tau_{\mu,0.05} = -2.86$. This is the
critical value when there is a drift $\mu$ in the regression, but no
deterministic trend
of the form $\lambda t$. (Note that Brooks has a typo in the critical value.) The observed value is $
\frac{\widehat \Psi}{SE(\widehat \Psi)} = -0.06$, which is too large to
reject the null hypothesis.
\item[(c)] There is insufficient evidence for rejecting the null, suggesting
that the data might be $I(1)$. To verify this, one could difference the
data and apply the DF test once more. If the original series is indeed
$I(1)$, this will most likely lead to a rejection of the null
hypothesis after differencing.
\item[(d)] The usual critical values do not apply since, due to the
non-stationarity of $y_t$, the ratio $\frac{\widehat \Psi}{SE(\widehat \Psi)}$
does not follow a standard $t$-distribution under the null hypothesis.
\end{enumerate}
\item[8.3]
\begin{enumerate}
\item[(a)] The DF test statistic in this case is
$\frac{\widehat \Psi}{SE(\widehat{\Psi})} = \frac{-0.52}{0.16} = -3.25$ which
is smaller than the critical value $\tau_{\mu,0.05} = -2.86$, hence there is
evidence that the data are actually generated by a stationary process.
\item[(b)] One might include a deterministic trend and apply the DF test once
more, since a mis-specification of this type (omitting it while it should be
there) can lead to over-rejection of the test (rejecting at a rate
larger than the nominal rejrection rate ($\alpha=0.05$ in this case)
under the null hypothesis).
\item[(c)] A solution to serial dependence in the residuals is to perform an
augmented Dickey Fuller test (including lagged values $\Delta y_{t-k}$ in
the regression, $k=1, \ldots, p$). The order $p$ can be selected automatically
in EViews.
\end{enumerate}
\item[8.6]
\begin{enumerate}
\item[(a)] These findings imply i) $r=0$, (ii) $r=1$, (iii) $r=2$, (iv) $r=3$ cointegrating relationships. $r=0$ corresponds to a nonstationary model, so one should estimate a VAR in first differences. $r=1$ and $r=2$ correspond to cointegration, so one should estimate an ECM. $r=3$ corresponds to a stationary model, so one should estimate a VAR in levels.
\item[(b)] We start from the first row. $H^0_0:r=0$ is rejected in favor of $H^0_{a}:r>0$ (the test statistic is larger than the critical value). Move on to row 2. Again, $H^1_0:r\leq 1$ is rejected in favor of $H^1_{a}:r>1$. Move on to row 3. This time, $H^2_0:r\leq 2$ is not rejected (the test statistic is smaller than the critical value). So we conclude that there are 2 cointegrating vectors.
\end{enumerate}

\end{itemize}
\newpage
\subsection*{Chapter 9 (p.\ 569--572): GARCH Models}
\begin{enumerate}
\item[9.1]
\begin{enumerate}
\item Some stylized features of returns data are: i) little to no autocorrelation, ii) heavy tails, iii) volatility clustering, iv) leverage effects. Linear (ARIMA) models can only be used to model the autocorrelation (and the heavy tails, if the error distribution is non-normal).
\item A GARCH(1,1) model (with normally distributed errors) can be used to model the volatility clustering. It also offers a partial explanation for the heavy tails, because it can be shown that volatility clustering leads to excess kurtosis in the unconditional distribution of returns.
\item It can be shown that a GARCH(1,1) model corresponds to an ARCH($\infty$) model, with a specific structure for the ARCH coefficients. It allows the researcher to model an autocorrelation structure in the squared returns for which a high-order ARCH model might be required, while requiring the estimation of only 3 parameters. Despite its parsimony, the GARCH(1,1) structure often proves sufficient to model the volatility clustering.
\item The simple GARCH(1,1) model cannot address the leverage effects found in financial returns. Extensions that allow for this include the GJR and EGARCH specifications. Furthermore, the GARCH model with normally distributed errors cannot account for all of the heavy-tailedness of the data; hence a non-normal error distribution may be warranted, such as the Student's $t$ or GED.
\item Daily returns typically have a mean close to zero, so $\mu\approx 0$. The estimates for the GARCH equation are often close to the stationarity boarder, e.g., $\alpha_1=0.05, \beta=0.94$. The unconditional volatility is given by $\sigma^2=\alpha_0/(1-\alpha_1-\beta)$. We expect an annualized volatility of perhaps 30\%, corresponding to a daily volatility of about 1.875\% $(30\%/\sqrt{256})$. Hence we expect that $\alpha_0\approx 0.000003515625$, so that $\sigma=\sqrt{0.000003515625/(1-0.99)}=0.01875$. If the returns are expressed in percent (i.e., they have been multiplied by 100), then $\alpha_0$ will be 10.000 times as large.
\item (has not been covered in class)
\item First, observe that $\hat{\alpha}_1+\hat{\beta}=1.1062>1$, so this model is non-stationary. In view of the next question, this appears to be a typo, so let's assume that $\hat{\beta}=0.811$, leading to a stationary model. Then
    \begin{align}
    \text{var}_{T}(y_{T+1})&=\sigma _{T+1}^{2}=\alpha_0+\alpha_1y_{T}^{2}+\beta \sigma _{T}^{2}\notag\\
    \text{var}_{T}(y_{T+k})&=\sigma
^{2}+(\alpha_1 +\beta )^{k-1}(\sigma _{T+1}^{2}-\sigma ^{2}),\quad k\geq 2\label{fore}
\end{align}
as given in the slides, and where we have put $T$ in place of $t$. Plugging in the estimates $\sigma _{T+1}^{2}$, $\hat{\alpha}_1$, $\hat{\beta}$, and $\hat{\sigma}=\sqrt{\hat{\alpha}_0/(1-\hat{\alpha}_1-\hat{\beta})}$, this can be used to forecast the conditional variance (and hence the volatility) $k$ days ahead.
\item  The forecast equation in the previous question is valid only if the model is stationary. Here, we have $\hat{\alpha_1}+\hat{\beta}=1.1051>1$, so this model is nonstationary. A simple derivation similar to the one given in the slides shows that
  \[\text{var}_{T}(y_{T+k})=\alpha_0\sum_{j=1}^{k-1}(\alpha_1+\beta)^{j-1}+(\alpha_1 +\beta )^{k-1}\sigma _{T+1}^{2}, \quad k\geq 2.
\]
If $(\alpha_1 +\beta )$, the volatility forecast will explode exponentially.
\end{enumerate}
\item[9.3]
\begin{enumerate}
\item The conditional variance $\sigma^2_{t+k}\equiv\mathrm{var}_t(y_{t+k})\equiv\mathrm{var}(y_{t+k}|\Omega_t)$ measures the variability of future returns, \emph{conditional} on the information available at time $t$. The unconditional variance $\sigma^2\equiv\mathbb{E}[\sigma^2_t]$, on the other hand, does not `use' the information whether period $t$ is in a low- or high volatility regime. It measures the \emph{average} variance.
A 1-day forecast relies strictly on the conditional variance. If the model is stationary, then according to \eqref{fore}, the multi period forecasts converge to the unconditional variance.
\item The regression in question is a regression of $y_t$ on just a constant. If heteroskedasticity is present but ignored, then $\mu$ will be estimated consistently, but its standard error will be off. Hence one should use heteroskedasticity consistent (`White') standard errors. Newey-West standard errors can also be used; they are consistent even if there is autocorrelation in addition to the heteroskedasticity. Hence their name, `HAC'.
\item Historical volatility is simple to compute, but reacts slowly to shocks and displays a `ghosting' feature. GARCH models are harder to estimate, but extremely successful in forecasting conditional volatility. EWMA is the method used by RiskMetrics. It is a special case of the GARCH model; all parameters are fixed, so no estimation is necessary. The model is thus very simple, but less flexible than a GARCH model. In particular, it is only suitable for daily data. Finally, implied volatility has the advantage that it is a forward-looking measure. However, the empirical evidence on the accuracy of implied versus statistical forecasting models is mixed, and some research suggests that implied volatility systematically over-estimates the true volatility of the underlying asset returns. This may arise from the use of an incorrect option pricing formula to obtain the implied volatility. As a result, other features of the data (such as excess kurtosis) are proxied for by the (implied) volatility.

\end{enumerate}
\end{enumerate}
\newpage
\section*{Chapter 11 (p.\ 654--656): Panel data}
 \begin{enumerate}
\item[10.1]
\begin{enumerate}
             \item[(a)]  A pooled regression refers to just stacking the time series for all individuals on top of one another and treating the data like one big sample, ignoring the panel character. The main advantage of the fixed effects estimator compared to a pooled regression is that the individual fixed effects allow us to account for the individual heterogeneity. As a consequence of this, the estimator is immune to omitted variable bias, as long as the omitted variables are time invariant, because their effect will be removed by the within transformation. Examples of such time invariant variables are company size, industry, CEO ability, etc. Note that the last one of these is unobservable, so we will never be able to obtain data on it; hence we cannot include it in a regression, and thus the fixed effects estimator is the only way to get rid of the omitted variable bias associated with it. A similar argument holds for the time effects; their inclusion rids us of all the effects of omitted variables that are constant across individuals, but vary over time (such as recessions).
                 \item[(c)] A balanced panel has the same number of observations for each individual, so that the total number of observations is $NT$. A typical example of an unbalanced panel is stock market data, because some companies may get delisted.
\end{enumerate}
\item[10.2]
\begin{enumerate}
   \item  The LSDV estimator can also be obtained by subtracting, from each outcome $y_i$ of individual $i$, the quantity $\bar y_{i \cdot}=\frac{1}{T}\sum_{t=1}^{T}y_{it}$, i.e., the mean over all observations for that individual. The same is done with each $x_{it}$. This yields
                                          \[
\ddot y_{it}=y_{it}-\bar y_{i\cdot}, \quad \ddot
x_{j,it}=x_{j,it}-\bar x_{j,i\cdot},
\]
This is called the \emph{within transformation}. The transformed model can be estimated by
OLS:
\[
\ddot y_{it}=\sum_{j=1}^l\ddot{x}_{j,it}\beta_j+\ddot
\nu_{it}.
\]
It can be shown that this is equivalent to including a dummy for each individual. The benefit is that it removes the hassle of having to estimate an extra $N$ parameters. For constructing tests and confidence intervals, we must only take care to use the proper degrees of freedom (because we implicitly still estimate the fixed effects). Eviews does that automatically. If the models includes time fixed effects as well, then the within transformation becomes
\[
\ddot y_{it}=y_{it}-\bar y_{i\cdot}-\bar y_{\cdot t}, \quad \ddot
x_{j,it}=x_{j,it}-\bar x_{j,i\cdot}-\bar x_{j,\cdot t},
\]
and we can again apply OLS to the transformed variables.
\item The RE estimator assumes that the individual heterogeneity is random, and INDEPENDENT of the regressors. The latter is unlikely to be true.
    \item If the individual effects are indeed independent of the regressors, then RE is better (consistent and efficient), FE is only consistent. If independence does not hold, then FE is consistent and efficient, but RE is neither. So FE is the safer choice. We can choose between them by means of a Hausman-Wu test.
 \end{enumerate}
\end{enumerate}
\end{document}
